\documentclass{article}

\usepackage{amsmath}
\usepackage{graphicx}
\usepackage{amsfonts}

\title{A Reformulation of QFT for UV-Finite Perturbation Theory}
\author{Jannis Naske}

\begin{document}

\maketitle

The usual way of quantizing fields is known to be ill-defined.
In some cases, this can be fixed through renormalization after formally deriving a perturbative expansion.
This paper introduces a reformulation of the quantization process, which may actually lead to a UV-finite perturbative expansion,
removing the need for renormalization. This reformulation is demonstrated in the $\phi^4$ theory.
The classical Hamiltonian formulation for this theory is given by
\begin{equation}\begin{gathered}
    H = \int d^nx \,\, \frac{1}{2} \pi^2(x) + \frac{1}{2} (\nabla\phi)^2(x) + \frac{1}{2} m^2\phi^2(x) + \frac{\lambda}{4!} \phi^4(x),\\
    \{\phi(x),\,\pi(y)\} = \delta(x - y), \quad \frac{\partial O(x)}{\partial t} = \{H,\,O(x)\},
\end{gathered}\end{equation}
where $O(x)$ is some function of $\phi(x)$ and $\pi(x)$. This is well defined, as there is one delta distribution per integral.
The equation of motion for $\phi$ is then
\begin{equation}\label{phi4}
    \frac{\partial^2}{\partial t^2} \phi(x) = \nabla^2\phi(x) - \frac{\lambda}{6} \phi^3(x).
\end{equation}
In the quantum theory, the fields are promoted to operators and the Poisson bracket to the commutator,
\begin{equation}
    [\phi(x),\,\pi(y)] = i\delta(x - y), \quad \frac{\partial O(x)}{\partial t} = i[H,\,O(x)].
\end{equation}
As an operator equation, (\ref{phi4}) still holds for $\phi(x)$. This is however no longer well defined,
as there may be additional delta functions per integral, as the field operators no longer commute.
This can be seen for example by Taylor expanding $\phi(x)$ in $t$.
The following proposition for a different formulation at least formally fixes this issue.
Assuming the relevant operators are integrals of polynomials of local operators, e.g. for $H$
\begin{equation}
    H = \int d^nx \,\, H(x),
\end{equation}
one introduces two commutators, one for local operators and one for global or smeared out operators:
\begin{equation}
    \begin{gathered}
        \left[ \int d^nx \,\, O_1(x), \, \int d^nx \,\, O_2(x) \right] := \int d^nx \,\, \left[ O_1(x), \, O_2(x) \right]_{loc},\\
        \left[ \phi(x), \, \pi(y) \right]_{loc} := i\delta_{x,y}.
    \end{gathered}
\end{equation}
Note that $\delta_{x,y}$ is now the Kronecker delta and not a distribution.
If one then smears out $\phi(x)$ and $\pi(x)$, this prescription reproduces (\ref{phi4}).
Now however, the Taylor expansion of $\phi$ in time does not produce additional delta distributions.
Another way of formally achieving the same thing is to convert the integral in $H$ to a sum and the delta distribution to a Kronecker delta,
\begin{equation}\begin{gathered}
    H = \sum_{x\in \mathbb{R}^n} \,\, \frac{1}{2} \pi^2(x) + \frac{1}{2} (\nabla\phi)^2(x) + \frac{1}{2} m^2\phi^2(x) + \frac{\lambda}{4!} \phi^4(x),\\
    \left[\phi(x),\,\pi(y)\right] = i\delta_{x, y}, \quad \frac{\partial O(x)}{\partial t} = \left[H,\,O(x)\right].
\end{gathered}\end{equation}
Now, in order to actually make sense of these prescriptions, one has to show that they can be defined rigorously.
One way of doing this is to regularize the Hamiltonian by considering a lattice approximation of the theory,
and then taking the limit as the lattice spacing tends to zero. For example, in one spacial dimension with a periodic boundary $\phi(N) = \phi(0)$ for simplicity, one gets
\begin{equation}
    H = \sum_{x = 0}^{N-1} \,\, \frac{1}{2} \pi^2(x) - \frac{1}{2} N^2 \phi(x)\left( \phi(x+1) - 2\phi(x) + \phi(x-1) \right) + \frac{1}{2} m^2\phi^2(x) + \frac{\lambda}{4!} \phi^4(x).
\end{equation}
While the spacial dimensions are discretized, time will remain continuous.\\\\
At the time of writing, it is not clear to me at all whether this reformulation can reproduce the usual renormalized perturbation theory,
or whether this even achieves anything at all. However, at least for the 1-loop Feynman diagram for the two particle interaction,
it seems that this method renders the loop UV-finite in the $N \rightarrow \infty$ limit,
and that the diagram actually converges to a non zero value, even in three spacial dimensions!
I will definitely give an update on this once I've made some major progress. I'm especially interested if
this method could be used in connection with constructive QFT.

\end{document}